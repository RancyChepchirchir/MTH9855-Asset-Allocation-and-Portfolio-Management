\documentclass[a4paper]{article}

%% Language and font encodings
\usepackage[english]{babel}
\usepackage[utf8x]{inputenc}
\usepackage[T1]{fontenc}

%% Sets page size and margins
\usepackage[a4paper,top=3cm,bottom=2cm,left=3cm,right=3cm,marginparwidth=1.75cm]{geometry}

%% Useful packages
\usepackage{amsmath}
\usepackage{amssymb}
\usepackage{graphicx}
\usepackage{enumerate}
\usepackage[colorinlistoftodos]{todonotes}
\usepackage[colorlinks=true, allcolors=blue]{hyperref}

\title{MTH 9855 Homework 1}
\author{Junliang Zhou}

\begin{document}
\maketitle

\section{Problem 1.1}

Prove that an agent is risk-averse, i.e. inequality
\[\mathbb{E}[u(w+\tilde{z})] \leq u(w+\mathbb{E}[\tilde{z}])\]
holds for all $w$ and $\tilde{z}$, if and only if $u()$ is concave.\newline

\textit{Proof.}\newline

Jensen's inequality states $\forall X$ as a random variable, $\phi()$ is a concave function if and only if
\[\phi(\mathbb{E}[X]) \geq \mathbb{E}[\phi(X)]\]

"$\Rightarrow$":\newline

Let $X=w+\tilde{z}$, then we have
\[\mathbb{E}[u(w+\tilde{z})] = \mathbb{E}[u(X)] \leq u(\mathbb{E}[X]) = u(w+\mathbb{E}[\tilde{z}])\]

by applying Jensen's inequality directly.\newline

"$\Leftarrow$":\newline

Let $X=w+\tilde{z}$, then we have
\[\mathbb{E}[u(X)] \leq u(\mathbb{E}[X])\]

holds for all $X$.\newline

Therefore, by applying Jensen's inequality $u()$ is a concave function.

\section{Problem 1.2}

Show that when $u(w) = -\text{exp}(-\kappa w)/\kappa$ and $\tilde{w}$ is normally distributed with mean $\mu$ and variance $\sigma^2$, then the Arrow-Pratt approximation is exact.\newline

\textit{Proof.}\newline

$u(\tilde{w}) = -\text{exp}(-\kappa \tilde{w})/\kappa$ is log-normally distributed, thus
\[\mathbb{E}[u(\tilde{w})]=-\frac{1}{\kappa}\mathbb{E}[\text{exp}(-\kappa \tilde{w})]=-\frac{1}{\kappa}\text{exp}(-\kappa\mu+\frac{1}{2}\kappa^2 \sigma^2)\]

For $u()$, we have
\[u'(\tilde{w})=\text{exp}(-\kappa \tilde{w})\]
\[u''(\tilde{w})=-\kappa\text{exp}(-\kappa \tilde{w})\]

As a result,
\[A(\tilde{w})=-\frac{u''(\tilde{w})}{u'(\tilde{w})}=\kappa\]

Since risk premium is defined as
\[\mathbb{E}[u(\tilde{w})]=u(\mu-\Pi)\]

we have
\[u(\mu-\Pi)=-\frac{1}{\kappa}\text{exp}(-\kappa(\mu-\frac{1}{2}\kappa\sigma^2))\]

Therefore,
\[\Pi=\frac{1}{2}\sigma^2 A(\tilde{w})\]

for normally distributed random variable $\tilde{w}$.

\section{Problem 1.3}

Prove the following three conditions are equivalent:
\begin{enumerate}[(a)]
\item Agent $v$ is more risk-averse than agent $u$;
\item For all $w$, $A_v(w) \geq A_u(w)$;
\item Function $v$ is a concave transformation of function $u$, meaning $\exists\phi, \phi'>0, \phi''<0$ s.t. $v(w)=\phi(u(w))$.
\end{enumerate}

\textit{Proof.}\newline

"(a)$\Rightarrow$(c)":\newline

Suppose $\exists\phi$ s.t. $v()=\phi(u())$. Since $u()$ and $v()=\phi(u())$ are both monotonically increasing functions and
\[v'(w)=\phi'(u(w))u'(w)\]

$\phi()$ is also a monotonically increasing function.\newline

Agent $v$ is more risk-averse than agent $u$ indicates that $\Pi_v>\Pi_u$. Then by monotonic property, 
\[v(w-\Pi_v)<\phi(u(w-\Pi_u))\]

Apply the definition of risk premium to the equation,
\[\mathbb{E}[v(w)]<\phi(\mathbb{E}[u(w)])\]
\[\mathbb{E}[\phi(u(w)]<\phi(\mathbb{E}[u(w)])\]

By Jensen's inequality $\phi()$ is a concave function.\newline

"(c)$\Rightarrow$(a)":\newline

Similarly, function $v$ is a concave transformation of function $u$ indicates that
\[\mathbb{E}[\phi(u(w)]<\phi(\mathbb{E}[u(w)])\]
\[\mathbb{E}[v(w)]<\phi(\mathbb{E}[u(w)])\]

Then by monotonic property, 
\[v(w-\Pi_v)<\phi(u(w-\Pi_u))\]

which shows $\Pi_v>\Pi_u$ since $u(), v()$ and $\phi()$ are all monotonically increasing functions.\newline

"(b)$\Rightarrow$(c)":\newline

From the definition of $A(w)$, we have
\[A_u(w)=-\frac{u''(w)}{u'(w)}\]
\[A_v(w)=-\frac{v''(w)}{v'(w)} = -\frac{\phi'(u(w))u''(w)+\phi''(u(w))(u'(w))^2}{\phi'(u(w))u'(w)}\]

Since $\phi'(u(w))>0$ as we shown above, $A_v(w) \geq A_u(w)$ is equivalent to
\[-\frac{\phi'(u(w))u''(w)}{\phi'(u(w))u'(w)} \geq -\frac{\phi'(u(w))u''(w)+\phi''(u(w))(u'(w))^2}{\phi'(u(w))u'(w)}\]

\[\frac{\phi''(u(w))(u'(w))^2}{\phi'(u(w))u'(w)} \leq 0\]

To satisfy the inequality, $\phi''(u(w)) \leq 0$ has to hold for all $w$. Therefore, $\phi()$ is a concave function.\newline

"(c)$\Rightarrow$(b)":\newline

Similarly, using the properties we proved before, we can construct the inequality,
\[-\frac{\phi'(u(w))u''(w)}{\phi'(u(w))u'(w)} \geq -\frac{\phi'(u(w))u''(w)+\phi''(u(w))(u'(w))^2}{\phi'(u(w))u'(w)}\]

holds for all $w$.\newline

Thus we have $A_v(w) \geq A_u(w)$ holds for all $w$.

\section{Problem 1.4}

Consider a function $v()$ such that $v(x)=a+bu(x)$ for all $x$, for some pair of scalars $a$ and $b$, where $b>0$. Show that a decision-maker with utility function $v()$ makes the same decisions and has the same certainty-equivalents as a decision maker with utility function $u()$.\newline

\textit{Proof.}\newline

We will prove the statement by showing they share the same risk premium $\Pi$.\newline

The risk premium of a decision maker with utility function $u()$ is defined as
\[\mathbb{E}[u(w+\tilde{z})]=u(w-\Pi)\]

where $\tilde{z}$ is a random variable with mean 0.\newline

As for a decision-maker with utility function $v()$,
\[\mathbb{E}[v(w+\tilde{z})]=\mathbb{E}[a+bu(w+\tilde{z})]=a+bu(w-\Pi)=v(w-\Pi)\]

Clearly, two decision-makers share the same risk premium $\Pi$.\newline

\end{document}