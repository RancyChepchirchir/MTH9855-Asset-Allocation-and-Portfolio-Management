\documentclass[a4paper]{article}

%% Language and font encodings
\usepackage[english]{babel}
\usepackage[utf8x]{inputenc}
\usepackage[T1]{fontenc}

%% Sets page size and margins
\usepackage[a4paper,top=3cm,bottom=2cm,left=3cm,right=3cm,marginparwidth=1.75cm]{geometry}

%% Useful packages
\usepackage{amsmath}
\usepackage{amssymb}
\usepackage{graphicx}
\usepackage{enumerate}
\usepackage[colorinlistoftodos]{todonotes}
\usepackage[colorlinks=true, allcolors=blue]{hyperref}

\title{MTH 9855 Homework 2}
\author{Junliang Zhou}

\begin{document}
\maketitle

\section{Problem 2.1}

Show that if 
\[U(\alpha p+(1-\alpha)p')=\alpha U(p)+(1-\alpha)U(p')\]
holds for every $p$, $p'$ and $\alpha$, then $U()$ must have an expected utility form.\newline

\textit{Proof.}\newline

Firstly we prove the linearity property by induction:\newline

Suppose
\[U(\sum_{i=1}^n {\alpha_i p_i}) = \sum_{i=1}^n {\alpha_i U(p_i)}\]

where
\[\sum_{i=1}^n {\alpha_i}=1\]

holds for all $\alpha_i\in[0,1]$ and $p_i\in\mathbb{R}$ when $n=k$.\newline

Clearly it holds for $n=1,2$ as
\[U(\alpha p+(1-\alpha)p')=\alpha U(p)+(1-\alpha)U(p')\]

When $n=k+1$, 
\[U(\sum_{i=1}^{k+1} {\alpha_i p_i}) = U(\sum_{i=1}^{k-1} {\alpha_i p_i}+(\alpha_k+\alpha_{k+1}) (\frac{\alpha_k}{\alpha_k+\alpha_{k+1}} p_k+\frac{\alpha_{k+1}}{\alpha_k+\alpha_{k+1}} p_{k+1}))\]

Let
\[\alpha'=\alpha_k+\alpha_{k+1}\]
\[p'=\frac{\alpha_k}{\alpha_k+\alpha_{k+1}} p_k+\frac{\alpha_{k+1}}{\alpha_k+\alpha_{k+1}} p_{k+1}\]

Then we have
\[U(\sum_{i=1}^{k+1} {\alpha_i p_i}) = U(\sum_{i=1}^{k-1} {\alpha_i p_i}+\alpha'p') = \sum_{i=1}^{k-1} {\alpha_i U(p_i)} + \alpha'U(p')\]

As
\[U(p')=U(\frac{\alpha_k}{\alpha_k+\alpha_{k+1}} p_k+\frac{\alpha_{k+1}}{\alpha_k+\alpha_{k+1}} p_{k+1})=\frac{\alpha_k}{\alpha_k+\alpha_{k+1}} U(p_k)+\frac{\alpha_{k+1}}{\alpha_k+\alpha_{k+1}} U(p_{k+1})\]

we can conclude from above that
\[U(\sum_{i=1}^n {\alpha_i p_i}) = \sum_{i=1}^n {\alpha_i U(p_i)}\]

where
\[\sum_{i=1}^n {\alpha_i}=1\]

holds for all $\alpha_i\in[0,1]$ and $p_i\in\mathbb{R}$ when $n=k+1$.\newline

By induction we can say that the equation above holds for all $n\in\mathbb{N}$.\newline

Secondly we show that $U()$ has an expected utility form:\newline

Defined on a finite reward space $\mathcal{X}$ and probability measure $\mathcal{P}$. A preference function $U:\mathcal{P}\rightarrow\mathbb{R}$ is said to have an expected utility form if there is some function $u:\mathcal{X}\rightarrow\mathbb{R}$ for all $p\in\mathcal{P}$ such that
\[U(p)=\mathbb{E}_p[u(x)]=\sum_{i=1}^n {p_i u(x_i)}\]

For any $p\in\mathcal{P}$, it can be viewed as a compound lottery which provides reward $x_i$ at state $i$ with probability $p_i$. It can be written as a linear combination of simple lotteries $x_i$ on probability measure $\mathcal{P}$,
\[p=\sum_{i=1}^n {p_i x_i}\]

Then by linearity of $U()$, we have
\[U(p)=U(\sum_{i=1}^n {p_i x_i})=\sum_{i=1}^n {p_i U(x_i)}\]

for all $p\in\mathcal{P}$.\newline

Therefore, $U()$ has an expected utility form.

\section{Problem 2.2}

Show that if the independence axiom is violated by a given decision-maker’s preferences, then there is a dutch book the decision-maker would agree to (i.e. an arbitrage they would be the source of), \textit{even if} the other three axioms are satisfied by the decision-maker’s preferences. Hint: In practice the way you construct a dutch book is by finding a sequence of trades that the agent would agree to, and which in the end, gives you an arbitrage. If they strictly prefer lottery $A$ to lottery $B$ then they would pay something (however small, doesn’t matter) to trade $B$ for $A$. If $A \sim B$, then they would exchange them for free, and so forth.\newline

\textit{Proof.}\newline

Assume the decision-maker has the preference that
\[L_1 \succeq L_2\]

where $L_1$ and $L_2$ are two different lotteries.\newline

Upon the violation of the independence axiom, we may have a lottery $L$ and $\alpha\in[0,1]$ such that
\[\alpha L + (1-\alpha)L_2 \succeq \alpha L + (1-\alpha) L_1\]

Suppose the decision-maker starts with a lottery $\alpha L + (1-\alpha) L_1$. I will offer a lottery $\alpha L + (1-\alpha)L_2$ in exchange for his. Since the new lottery is more preferred by him, he is willingly to pay a positive amount of cash $C_1$ for the trade. Then I will offer a compound lottery with probability $\alpha$ to become $L$ and $(1-\alpha)$ to become $L_2$. Since the new compound lottery is indifference to $\alpha L + (1-\alpha)L_2$, he is willingly to exchange freely. After that, if the compound lottery turns out to be $L_2$, I will offer $L_1$ instead of $L_2$. Since $L_1$ is more preferred than $L_2$, he is still willingly to trade with a positive amount of cash $C_2$.\newline

After this process, the decision-maker will still have lottery $\alpha L + (1-\alpha) L_1$, while I will get back lottery $\alpha L + (1-\alpha) L_2$, which I initially offers, and close my position. Therefore, I start with no initial capital and get an positive expected cash amount of $C_1+(1-\alpha)C_2$.

\section{Problem 2.3}

Show that, given a complete and transitive preference relation on $\mathcal{P}$, if $U$ is an expected utility representation of $\succeq$, then $\succeq$ must satisfy continuity and independence.\newline

\textit{Proof.}\newline

"Continuity":\newline

Suppose a preference $p \succeq p' \succeq p''$ which is complete and transitive, while $U()$ is is an expected utility representation of $\succeq$, then we have
\[U(p)\geq U(p')\geq U(p'')\]

Let $\alpha p+(1-\alpha)p''$ be a compound lottery with $\alpha\in[0,1]$, then by the property of expected utility,
\[U(\alpha p+(1-\alpha)p'')=\alpha U(p)+(1-\alpha)U(p'')\]

Now let
\[\alpha=\frac{U(p')-U(p'')}{U(p)-U(p'')}\]

which belongs to the domain of $\alpha$ clearly.\newline

Substitute $\alpha$ back we have
\[U(\alpha p+(1-\alpha)p'')=U(p')\]

which implies that
\[\alpha p+(1-\alpha)p'' \sim p'\]

"Independence":\newline

Suppose a preference $p_1 \succeq p_2$ which is complete and transitive, while $U()$ is is an expected utility representation of $\succeq$, then we have
\[U(p_1)\geq U(p_2)\]

Let $\alpha p+(1-\alpha)p_1$ and $\alpha p+(1-\alpha)p_2$ be two compound lotteries with $\alpha\in[0,1]$, then by the property of expected utility,
\[U(\alpha p+(1-\alpha)p_1)=\alpha U(p)+(1-\alpha)U(p_1)\]
\[U(\alpha p+(1-\alpha)p_2)=\alpha U(p)+(1-\alpha)U(p_2)\]

It is obvious that
\[U(\alpha p+(1-\alpha)p_1) \geq U(\alpha p+(1-\alpha)p_2)\]

which implies that
\[\alpha p+(1-\alpha)p_1 \succeq \alpha p+(1-\alpha)p_2\]

\end{document}